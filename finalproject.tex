\documentclass{article}
\usepackage[utf8]{inputenc}
\usepackage[english]{babel}
%notas para resolver: no me gusta que quede con tantos márgenes. Para el jueves 24 poner unos márgenes más pequeños. 

\title{Proyecto del curso: Latex y Git aplicado a la investigación científica}
\author{Elena Campoy}
\date{October 2019}

\begin{document}

\maketitle

\section{Introduction}
A chromosomal inversion is a type of structural variant that consists of a change in orientation of a segment of DNA \cite{puig_human_2015, giner-delgado_evolutionary_2019}. In spite of being the first type of genetic variant to be estudied, some of their characteristics make its detection and study complicated. For example, they do not mean a gain or lose of genetic information. Therefore, they still remain as one of the most difficult classes of genetic variation to identify and characterize \cite{puig_determining_2019} and a bit of effort has to be done to fully understand their role in genetic variation. 
\
The main objective of this article is to summarize the most important information of inversions and the recents advances in their study. 
%aqui puedo hablar un poco de drosophila
\section{Human inversions}
\subsection{Mechanism of formation}

\section{How to detect an inversion}
\section{What does an inversion do?}
\newpage
\bibliographystyle{plain}
\bibliography{project.bib}
\end{document}
